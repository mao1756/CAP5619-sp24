%%%%%%%%%%%%%%%%%%%%%%%%%%%%%%%%%%%%%%%%%
% fphw Assignment
% LaTeX Template
% Version 1.0 (27/04/2019)
%
% This template originates from:
% https://www.LaTeXTemplates.com
%
% Authors:
% Class by Felipe Portales-Oliva (f.portales.oliva@gmail.com) with template 
% content and modifications by Vel (vel@LaTeXTemplates.com)
%
% Template (this file) License:
% CC BY-NC-SA 3.0 (http://creativecommons.org/licenses/by-nc-sa/3.0/)
%
%%%%%%%%%%%%%%%%%%%%%%%%%%%%%%%%%%%%%%%%%

%----------------------------------------------------------------------------------------
%	PACKAGES AND OTHER DOCUMENT CONFIGURATIONS
%----------------------------------------------------------------------------------------

\documentclass[
	12pt, % Default font size, values between 10pt-12pt are allowed
	%letterpaper, % Uncomment for US letter paper size
	%spanish, % Uncomment for Spanish
]{../Template/fphw}

% Template-specific packages
\usepackage[utf8]{inputenc} % Required for inputting international characters
\usepackage[T1]{fontenc} % Output font encoding for international characters
\usepackage{mathpazo} % Use the Palatino font

\usepackage{graphicx} % Required for including images

\usepackage{booktabs} % Required for better horizontal rules in tables

\usepackage{listings} % Required for insertion of code

\usepackage{enumerate} % To modify the enumerate environment

% Additional packages needed
\usepackage{amsmath}
\usepackage{enumitem}
\usepackage{mwe}
\usepackage{comment}

%----------------------------------------------------------------------------------------
%	ASSIGNMENT INFORMATION
%----------------------------------------------------------------------------------------

\title{Term Project: Proposal} % Assignment title

\author{Mao Nishino} % Student name

\date{April 11th, 2024} % Due date

\institute{Florida State University \\ Department of Computer Science} % Institute or school name

\class{Deep and Reinforcement Learning Fundamentals (CAP5619-0001.sp24)} % Course or class name

\professor{Dr. Xiuwen Liu} % Professor or teacher in charge of the assignment

%----------------------------------------------------------------------------------------

\begin{document}

\maketitle % Output the assignment title, created automatically using the information in the custom commands above

%----------------------------------------------------------------------------------------
%	ASSIGNMENT CONTENT
%----------------------------------------------------------------------------------------

\section*{Question 1}

\begin{problem}
 For a fully connected multiple-layer perceptron neural network described in Algorithms 6.3 and
6.4, where the L2 regularization of all the weights and biases is used as the regularization term (W(q)).
\begin{enumerate}[label=(\arabic*)]
    \item Give the number of addition operations, number of multiplication operations, and the number of activation
function calculations for Algorithm 6.3. You need to express your answers in terms of columns and rows of the
weight matrices and the length of bias vectors. Furthermore, we assume the square of x is done by x * x.
    \item As in (1), give the numbers of addition operations, multiplication operations, and activation function
calculations for Algorithm 6.4.
Note that the addition and multiplication operations needed for activation function calculations are included implicitly
in the number of activation function calculations and should not be included as additional addition and multiplication
operations.

\end{enumerate}
\end{problem}

%------------------------------------------------

\subsection*{Answer}


%----------------------------------------------------------------------------------------

\section*{Question 2}

\begin{problem}
	How much wood would a woodchuck chuck if a woodchuck could chuck wood?
	
	\medskip
	
	\begin{enumerate}[label = (\alph*)] % Sub-questions styled as italic letters
		\item Suppose ``chuck" implies throwing.
		\item Suppose ``chuck" implies vomiting.
	\end{enumerate}
\end{problem}

%------------------------------------------------

\subsection*{Answer}

\begin{enumerate}[label = (\alph*)] % Sub-questions styled as italic letters
	\item According to the Associated Press (1988), a New York Fish and Wildlife technician named Richard Thomas calculated the volume of dirt in a typical 25--30 foot (7.6--9.1 m) long woodchuck burrow and had determined that if the woodchuck had moved an equivalent volume of wood, it could move ``about \textbf{700 pounds (320 kg)} on a good day, with the wind at his back".
    
	\item A woodchuck can ingest 361.92 cm\textsuperscript{3} (22.09 cu in) of wood per day. Assuming immediate expulsion on ingestion with a 5\% retainment rate, a woodchuck could chuck \textbf{343.82 cm\textsuperscript{3}} of wood per day.
\end{enumerate}

%----------------------------------------------------------------------------------------

\section*{Question 3}

\begin{problem}
	Identify the author of Equation \ref{eq:bayes} below and briefly describe it in Latin.
	
	\medskip
	
	\begin{equation}\label{eq:bayes}
		P(A|B) = \frac{P(B|A)P(A)}{P(B)}
	\end{equation}
	
	\smallskip
\end{problem}

%------------------------------------------------

\subsection*{Answer} 

Lorem ipsum dolor sit amet, consectetur adipiscing elit. Praesent porttitor arcu luctus, imperdiet urna iaculis, mattis eros. Pellentesque iaculis odio vel nisl ullamcorper, nec faucibus ipsum molestie. Sed dictum nisl non aliquet porttitor. Etiam vulputate arcu dignissim, finibus sem et, viverra nisl. Aenean luctus congue massa, ut laoreet metus ornare in. Nunc fermentum nisi imperdiet lectus tincidunt vestibulum at ac elit. Nulla mattis nisl eu malesuada suscipit.

%----------------------------------------------------------------------------------------

\section*{Question 4 (bonus marks)}

\begin{problem}
	The table below shows the nutritional consistencies of two sausage types. Explain their relative differences given what you know about daily adult nutritional recommendations.
	
	\bigskip
    
	\begin{center}
		\begin{tabular}{l l l}
			\toprule
			\textit{Per 50g} & Pork & Soy \\
			\midrule
			Energy & 760kJ & 538kJ\\
			Protein & 7.0g & 9.3g\\
			Carbohydrate & 0.0g & 4.9g\\
			Fat & 16.8g & 9.1g\\
			Sodium & 0.4g & 0.4g\\
			Fibre & 0.0g & 1.4g\\
			\bottomrule
		\end{tabular}
	\end{center}
	
	\medskip
\end{problem}

%------------------------------------------------

\subsection*{Answer}

Lorem ipsum dolor sit amet, consectetur adipiscing elit. Praesent porttitor arcu luctus, imperdiet urna iaculis, mattis eros. Pellentesque iaculis odio vel nisl ullamcorper, nec faucibus ipsum molestie. Sed dictum nisl non aliquet porttitor. Etiam vulputate arcu dignissim, finibus sem et, viverra nisl. Aenean luctus congue massa, ut laoreet metus ornare in. Nunc fermentum nisi imperdiet lectus tincidunt vestibulum at ac elit. Nulla mattis nisl eu malesuada suscipit.

%----------------------------------------------------------------------------------------

\section*{Question 5 (bonus marks)}

\begin{problem}
	\lstinputlisting[
		caption=Luftballons Perl Script, % Caption above the listing
		label=lst:luftballons, % Label for referencing this listing
		language=Perl, % Use Perl functions/syntax highlighting
		frame=single, % Frame around the code listing
		showstringspaces=false, % Don't put marks in string spaces
		numbers=left, % Line numbers on left
		numberstyle=\tiny, % Line numbers styling
	]{}
	
	\begin{enumerate}
		\item How many luftballons will be output by the Listing \ref{lst:luftballons} above?
		\item Identify the regular expression in Listing \ref{lst:luftballons} and explain how it relates to the anti-war sentiments found in the rest of the script.
	\end{enumerate}

\end{problem}

%------------------------------------------------

\subsection*{Answer}

\begin{enumerate}
	\item 99 luftballons.
	\item Lorem ipsum dolor sit amet, consectetur adipiscing elit. Praesent porttitor arcu luctus, imperdiet urna iaculis, mattis eros. Pellentesque iaculis odio vel nisl ullamcorper, nec faucibus ipsum molestie. Sed dictum nisl non aliquet porttitor. Etiam vulputate arcu dignissim, finibus sem et, viverra nisl. Aenean luctus congue massa, ut laoreet metus ornare in. Nunc fermentum nisi imperdiet lectus tincidunt vestibulum at ac elit. Nulla mattis nisl eu malesuada suscipit.
\end{enumerate}

%----------------------------------------------------------------------------------------

\end{document}
