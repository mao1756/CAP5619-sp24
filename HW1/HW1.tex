%%%%%%%%%%%%%%%%%%%%%%%%%%%%%%%%%%%%%%%%%
% fphw Assignment
% LaTeX Template
% Version 1.0 (27/04/2019)
%
% This template originates from:
% https://www.LaTeXTemplates.com
%
% Authors:
% Class by Felipe Portales-Oliva (f.portales.oliva@gmail.com) with template 
% content and modifications by Vel (vel@LaTeXTemplates.com)
%
% Template (this file) License:
% CC BY-NC-SA 3.0 (http://creativecommons.org/licenses/by-nc-sa/3.0/)
%
%%%%%%%%%%%%%%%%%%%%%%%%%%%%%%%%%%%%%%%%%

%----------------------------------------------------------------------------------------
%	PACKAGES AND OTHER DOCUMENT CONFIGURATIONS
%----------------------------------------------------------------------------------------

\documentclass[
	12pt, % Default font size, values between 10pt-12pt are allowed
	%letterpaper, % Uncomment for US letter paper size
	%spanish, % Uncomment for Spanish
]{../Template/fphw}

% Template-specific packages
\usepackage[utf8]{inputenc} % Required for inputting international characters
\usepackage[T1]{fontenc} % Output font encoding for international characters
\usepackage{mathpazo} % Use the Palatino font

\usepackage{graphicx} % Required for including images

\usepackage{booktabs} % Required for better horizontal rules in tables

\usepackage{listings} % Required for insertion of code

\usepackage{enumerate} % To modify the enumerate environment


% Additional packages needed
\usepackage{amsmath}
\usepackage{enumitem}
\usepackage{mwe}

%----------------------------------------------------------------------------------------
%	ASSIGNMENT INFORMATION
%----------------------------------------------------------------------------------------

\title{Homework \#1} % Assignment title

\author{Mao Nishino} % Student name

\date{February 13th, 2024} % Due date

\institute{Florida State University \\ Department of Computer Science} % Institute or school name

\class{Deep and Reinforcement Learning Fundamentals (CAP5619-0001.sp24)} % Course or class name

\professor{Dr. Xiuwen Liu} % Professor or teacher in charge of the assignment

%----------------------------------------------------------------------------------------

\begin{document}

\maketitle % Output the assignment title, created automatically using the information in the custom commands above

%----------------------------------------------------------------------------------------
%	ASSIGNMENT CONTENT
%----------------------------------------------------------------------------------------

\section*{Question 1}

\begin{problem}
	As neural networks are typically trained using (stochastic) gradient descent optimization
algorithms, properties of the activation functions affect the learning. Here we divide the domain of an activation
function into: 1) fast learning region if the magnitude of the gradient is larger than 0.99, 2) active learning region if
the magnitude of the gradient is between 0.01 and 0.99 (inclusive), 3) slow learning region if the magnitude of the
gradient is larger than 0 but smaller than 0.01, and 4) inactive learning region if the magnitude of the gradient is 0.
For each of the following activation functions, \textbf{plot} its gradient in the range from -5 to 5 of the input and then \textbf{list}
the four types of regions. If the gradient is not well defined for an input value, indicate so and then use any reasonable
value.

\begin{enumerate}[label = (\arabic*)]
    \item Rectified linear unit \( f(z) = \begin{cases} z & \text{if } z \geq 0 \\ 0 & \text{otherwise} \end{cases} \)
    \item Logistic sigmoid activation function \( f(z) = \sigma(z) = \frac{1}{1+e^{-z}} \) 
    \item Piece-wise linear unit \( f(z) = \begin{cases} 0.1z+0.9 & \text{if } z> 1 \\ z & \text{if }1\geq z\geq -1 \\ 0.1z-0.9 & \text{Otherwise} \end{cases} \) 
    \item Swish \(f(z) =  z\sigma(2.5z)\), where $\sigma(z) = \frac{1}{1+e^{-z}}$.
    \item Exponential Linear Unit (ELU) \( f(z) = \begin{cases} z & \text{if } z \geq 0 \\ 0.05(e^z-1) & \text{otherwise} \end{cases} \) (Note here is a special case of the general ELU function with $\alpha=0.05$.)
\end{enumerate}

\end{problem}

%------------------------------------------------

\subsection*{Answer}
We will first find the gradients analytically and then show the plot (See Figure \ref{fig:gradients_q1}) and the list (See Table \ref{tab:activation_region_q1}) required.

\begin{enumerate}[label = (\arabic*)]
\item The gradient of the ReLU function is \( f'(z) = \begin{cases} 1 & \text{if } z > 0 \\ \text{undefined} & z=0\\ 0 & z<0 \end{cases}\). In Figure \ref{fig:gradients_q1}, the value of gradient at $z=0$ is set to be $0$.  
\item By the Chain Rule, the gradient of the logistic sigmoid activation function is $f'(z) = -\frac{1}{(1+e^{-z})^2}\cdot (1+e^{-z})' = -\frac{1}{(1+e^{-z})^2} \cdot (-e^{-z}) = \frac{e^{-z}}{(1+e^{-z})^2}$.
\item The gradient of the piece-wise linear unit is \( f'(z) = \begin{cases} 1 & \text{if } 0<z<1 \\ 0.1 & z>1, z<-1 \\ \text{undefined} & z=-1,1 \end{cases}\). In Figure \ref{fig:gradients_q1}, we will use $1$ for the undefined gradients.
\item In an explicit form, Swish is $f(z) =\frac{z}{1+e^{-2.5z}}$. Therefore, by the Quotient Rule, we have 
\begin{align*}f'(z) &= \frac{1\cdot (1+e^{-2.5z})-z\cdot (-2.5)e^{-2.5z}}{(1+e^{-2.5z})^2} \\ &= \frac{1+e^{-2.5z}+2.5ze^{-2.5z}}{(1+e^{-2.5z})^2}\end{align*}
\item The gradient of the ELU function is \( f'(z) = \begin{cases} 1 & \text{if } z > 0 \\ \text{undefined} & z=0\\ 0.05 & z<0 \end{cases}\). In Figure \ref{fig:gradients_q1}, the value of gradient at $z=0$ is set to be $1$. 

\end{enumerate}

\begin{figure}[!htbp]
    \centering
    \includegraphics{example-image}
    \caption{Caption}
    \label{fig:gradients_q1}
\end{figure}

\begin{table}[!htbp]
    \centering
    \begin{tabular}{c|c}
         &  \\
         & 
    \end{tabular}
    \caption{Caption}
    \label{tab:activation_region_q1}
\end{table}

%----------------------------------------------------------------------------------------

\section*{Question 2}

\begin{problem}
	How much wood would a woodchuck chuck if a woodchuck could chuck wood?
	
	\medskip
	
	\begin{enumerate}[label = (\arabic*)]
		\item Suppose ``chuck" implies throwing.
		\item Suppose ``chuck" implies vomiting.
	\end{enumerate}
\end{problem}

%------------------------------------------------

\subsection*{Answer}

\begin{enumerate}[label = (\arabic*)]
	\item According to the Associated Press (1988), a New York Fish and Wildlife technician named Richard Thomas calculated the volume of dirt in a typical 25--30 foot (7.6--9.1 m) long woodchuck burrow and had determined that if the woodchuck had moved an equivalent volume of wood, it could move ``about \textbf{700 pounds (320 kg)} on a good day, with the wind at his back".
    
	\item A woodchuck can ingest 361.92 cm\textsuperscript{3} (22.09 cu in) of wood per day. Assuming immediate expulsion on ingestion with a 5\% retainment rate, a woodchuck could chuck \textbf{343.82 cm\textsuperscript{3}} of wood per day.
\end{enumerate}

%----------------------------------------------------------------------------------------

\section*{Question 3}

\begin{problem}
	Identify the author of Equation \ref{eq:bayes} below and briefly describe it in Latin.
	
	\medskip
	
	\begin{equation}\label{eq:bayes}
		P(A|B) = \frac{P(B|A)P(A)}{P(B)}
	\end{equation}
	
	\smallskip
\end{problem}

%------------------------------------------------

\subsection*{Answer} 

Lorem ipsum dolor sit amet, consectetur adipiscing elit. Praesent porttitor arcu luctus, imperdiet urna iaculis, mattis eros. Pellentesque iaculis odio vel nisl ullamcorper, nec faucibus ipsum molestie. Sed dictum nisl non aliquet porttitor. Etiam vulputate arcu dignissim, finibus sem et, viverra nisl. Aenean luctus congue massa, ut laoreet metus ornare in. Nunc fermentum nisi imperdiet lectus tincidunt vestibulum at ac elit. Nulla mattis nisl eu malesuada suscipit.

%----------------------------------------------------------------------------------------

\section*{Question 4 (bonus marks)}

\begin{problem}
	The table below shows the nutritional consistencies of two sausage types. Explain their relative differences given what you know about daily adult nutritional recommendations.
	
	\bigskip
    
	\begin{center}
		\begin{tabular}{l l l}
			\toprule
			\textit{Per 50g} & Pork & Soy \\
			\midrule
			Energy & 760kJ & 538kJ\\
			Protein & 7.0g & 9.3g\\
			Carbohydrate & 0.0g & 4.9g\\
			Fat & 16.8g & 9.1g\\
			Sodium & 0.4g & 0.4g\\
			Fibre & 0.0g & 1.4g\\
			\bottomrule
		\end{tabular}
	\end{center}
	
	\medskip
\end{problem}

%------------------------------------------------

\subsection*{Answer}

Lorem ipsum dolor sit amet, consectetur adipiscing elit. Praesent porttitor arcu luctus, imperdiet urna iaculis, mattis eros. Pellentesque iaculis odio vel nisl ullamcorper, nec faucibus ipsum molestie. Sed dictum nisl non aliquet porttitor. Etiam vulputate arcu dignissim, finibus sem et, viverra nisl. Aenean luctus congue massa, ut laoreet metus ornare in. Nunc fermentum nisi imperdiet lectus tincidunt vestibulum at ac elit. Nulla mattis nisl eu malesuada suscipit.

%----------------------------------------------------------------------------------------

\section*{Question 5 (bonus marks)}

\begin{problem}
	\lstinputlisting[
		caption=Luftballons Perl Script, % Caption above the listing
		label=lst:luftballons, % Label for referencing this listing
		language=Perl, % Use Perl functions/syntax highlighting
		frame=single, % Frame around the code listing
		showstringspaces=false, % Don't put marks in string spaces
		numbers=left, % Line numbers on left
		numberstyle=\tiny, % Line numbers styling
	]{}
	
	\begin{enumerate}
		\item How many luftballons will be output by the Listing \ref{lst:luftballons} above?
		\item Identify the regular expression in Listing \ref{lst:luftballons} and explain how it relates to the anti-war sentiments found in the rest of the script.
	\end{enumerate}

\end{problem}

%------------------------------------------------

\subsection*{Answer}

\begin{enumerate}
	\item 99 luftballons.
	\item Lorem ipsum dolor sit amet, consectetur adipiscing elit. Praesent porttitor arcu luctus, imperdiet urna iaculis, mattis eros. Pellentesque iaculis odio vel nisl ullamcorper, nec faucibus ipsum molestie. Sed dictum nisl non aliquet porttitor. Etiam vulputate arcu dignissim, finibus sem et, viverra nisl. Aenean luctus congue massa, ut laoreet metus ornare in. Nunc fermentum nisi imperdiet lectus tincidunt vestibulum at ac elit. Nulla mattis nisl eu malesuada suscipit.
\end{enumerate}

%----------------------------------------------------------------------------------------

\end{document}
